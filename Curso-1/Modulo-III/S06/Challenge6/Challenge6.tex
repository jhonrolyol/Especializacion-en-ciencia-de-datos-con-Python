% Preamble
	\documentclass[11pt]{article}
	% Usepackage
	\usepackage[spanish]{babel} % Language
	\usepackage{amsmath} % Maths
	\usepackage{xcolor} % Colors
	\usepackage[left = 2.54cm, right = 2.54cm, top = 2.54cm, bottom = 2.54cm]{geometry}
	% Title
	\title{\textbf{CHALLENGE 6}} % Title
	\author{Jhon Roly Ordoñez Leon} % Name
	\date{\today} % Date
% Body
\begin{document}
	% Calling title
	\maketitle
	% Table of contents
	% Section 1
	\section{Pregunta 1}
	Describe un caso donde apliques el ciclo básico de un análisis estadístico (Similar al caso de las elecciones o modelos en Banca). Sé lo más detallado posible (identificando los elementos de la estadística y/o utilizando gráficos de apoyo)
		\begin{center}
			\textbf{Solución}
		\end{center}
	Supongamos que se desea conocer la calificación promedio de todos los ingresantes en las universidades del Perú. Entonces, la  \textcolor{blue}{población}
	estaría formada por las calificaciones de todos los ingresantes en las universidades del Perú (la calificación promedio de todos los ingresantes en las universidades del Perú sería un parámetro); la \textcolor{blue}{muestra} en este caso particular podría ser las calificaciones 
	de los ingresantes en la Universidad Nacional de Ingeniería (la calificación promedio de todos los ingresantes en la Universidad Nacional de Ingeniería sería un estadígrafo).
	
	% Section 2
	\section{Pregunta 2}
	En una urna hay 4 bolas rojas, 3 bolas verdes y 2 bolas azules. Si se extrae una bola al azar, ¿cuál es la probabilidad de que sea verde? 
		\begin{center}
			\textbf{Solución}
		\end{center}
	
	Sea $R =$ Rojo, $V =$ Verde, $A=$ Azul, entonces
	\begin{equation}
		S = \left\{  4R, 3V, 2A \right\}
	\end{equation}
	Donde, nos pide 
	\begin{equation}
		A = \left\{  \text{Obtener una bola de color verde} \right\}
	\end{equation}
	Por lo tanto, la probabilidad de que sea verde es:
	\begin{equation}
		P(A ) = \dfrac{3}{9} = 0.33
	\end{equation}
	
	% Section 3
	\section{Pregunta 3}
	Si una baraja de cartas contiene 52 cartas, ¿cuál es la probabilidad de sacar una carta de corazones al azar?
		\begin{center}
			\textbf{Solución}
		\end{center}
		
	Sea $T =$ Trebol, $C =$ Corazones, $E=$ Espada, $D=$ Diamante, entonces
	\begin{equation}
		S = \left\{  13T, 13C, 13E, 13D \right\}
	\end{equation}
	Donde, nos pide 
	\begin{equation}
		A = \left\{ \text{Obtener una carta de corazones} \right\}
	\end{equation}
	Por lo tanto, la probabilidad de sacar una carta de corazones es:
	\begin{equation}
		P(A ) = \dfrac{13}{52} = 0.25
	\end{equation}
	
	% Section 4
	\section{Pregunta 4}
	Si lanzamos dos dados justos, ¿cuál es la probabilidad de que la suma de los puntos sea 7? (Sugerencia: Enlista los posibles resultados de lanzar dos dados para obtener tu espacio muestral y cuenta cuántos resultados suman 7)
		\begin{center}
			\textbf{Solución}
		\end{center}
	Listamos los posibles resultados de lanzar dos dados 
		\begin{equation}
			S = \{ 2, 3, 4, 5, 6, \textcolor{blue}{7}, 3, 4, 5, 6, \textcolor{blue}{7}, 8, 4, 5, 6, \textcolor{blue}{7}, 8, 9, 5,  6, \textcolor{blue}{7}, 8, 9, 10, 6, \textcolor{blue}{7}, 8, 9, 10, 11, \textcolor{blue}{7}, 8, 9, 10, 11, 12  \}
		\end{equation}
	Donde, nos pide 
	\begin{equation}
		A = \left\{ \text{Obtener una suma de 7} \right\}
	\end{equation}
	Por lo tanto, la probabilidad de obtener una suma de 7 es:
	\begin{equation}
		P(A ) = \dfrac{6}{36} = 0.16
	\end{equation}

	% Section 5
	\section{Pregunta 5}
	Si se lanzan dos monedas (suponer que son monedas no trucadas), ¿cuál es la probabilidad de obtener al menos una cara? (Sugerencia: Enlista los posibles resultados de lanzar dos monedas: Cara-cara, cara-sello,etc)
		\begin{center}
			\textbf{Solución}
		\end{center}
    Listamos los posibles resultados de lanzar dos monedas
		\begin{equation}
			S = \{ CC, CS, SC, SS \}
		\end{equation}
	Donde,  nos pide 
	\begin{equation}
		A = \left\{ \text{Obtener al menos una cara} \right\}
	\end{equation}
	Por lo tanto, la probabilidad de obtener al menos una cara es:
	\begin{equation}
		P(A ) = \dfrac{3}{4} = 0.75
	\end{equation}

	% Section 6
	\section{Pregunta 6}
	Una urna contiene 5 bolas rojas y 3 bolas verdes. Si se extraen dos bolas al azar sin reemplazo, ¿cuál es la probabilidad de que ambas sean rojas? (Sugerencia: Calcula la probabilidad de que la primera bola sea roja y multiplicarla con la probabilidad de que la segunda bola sea roja, recuerda que para la segunda selección hay menos bolas)
		\begin{center}
			\textbf{Solución}
		\end{center}
	Sea $R = $ Roja y $V = $ Verde, entonces el espacio muestral sería:
	\begin{equation}
		\Omega = \left \{RR,RV,VR,VV\right\}
	\end{equation}
 	Donde, nos pide
 	\begin{equation}
 		A = \left \{\text{Obtener que ambas bolas sean Rojas}\right\}
 	\end{equation}
	Probabilidad de que la primera bola sea roja es:
	\begin{equation}
		P(A ) = \dfrac{5}{8}
	\end{equation}
	 Probabilidad de que la segunda bola sea roja es:
	\begin{equation}
		P(A ) = \dfrac{4}{7}
	\end{equation}
	Por lo tanto, la probabilidad de que ambas bolas sean rojas es:
	\begin{equation}
		P(A ) = \dfrac{5}{8}\times\dfrac{4}{7}  = \dfrac{5}{14}
	\end{equation}
\end{document}




