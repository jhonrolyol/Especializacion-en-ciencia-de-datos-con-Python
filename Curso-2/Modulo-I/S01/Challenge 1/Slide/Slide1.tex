% Preámbulo
    \documentclass[aspectratio=169]{beamer}
    \usepackage[utf8]{inputenc}
    % Paquetes
        \usepackage{multimedia}
        \usepackage{hyperref}
        \usepackage{xcolor}
        \usepackage{graphicx}
        \usepackage{lipsum}
        \usepackage{pgfpages}
    % Tema 
        \usetheme[progressbar=frametitle]{metropolis}
        \setbeamertemplate{frame numbering}[fraction]
        \definecolor{primary}{RGB}{245, 10, 10}
        \setbeamercolor{palette primary}{bg=white, fg=black}
        \setbeamercolor{background canvas}{parent=palette primary}
        \setbeamercolor{normal text}{fg=black}
        \setbeamercolor{progress bar}{use=palette primary,fg=primary}
    % Título de la página
        \title{CHALLENGE 1}
        \subtitle{Módulo Estadística Descriptiva con Python}
    	\institute[Proyecto Final]{
    		{\large
    			\begin{center}
    				\textbf{GRUPO ESTUDIO}
    			\end{center}
    		}
    		\flushleft{ \textbf{INTEGRANTES.-} }
    		\begin{itemize}
    			\item Alonso Macha, Alex Daniel
    			\item Lopez Rengifo, Antonio Jose
    			\item Llanque Zapana, Luis Fernando
    			\item Miranda Hankgo, Ivan Cesar
    			\item Ordoñez Leon, Jhon Roly (Representante)
    		\end{itemize} 
    	}
        \date{}
% Cuerpo del documento
\begin{document}
    % Título
    \begin{frame}
        \titlepage
    \end{frame}
    % Contenido
    \begin{frame}{Contenido}
        \tableofcontents
    \end{frame}
    % Sección 1
    \section{1.- Tema}
        \begin{frame}{Tema}
            \begin{center}
            	\textbf{“Precios de la vivienda - Técnicas Avanzadas de Regresión”}
            \end{center}
            Partiendo de la necesidad de una inmobiliaria por conocer el comportamiento
            del precio de  una vivienda según las variables explicativas que describen
            (casi) todos los aspectos de las casas residenciales en Ames, Iowa, 
            esta competencia lo desafía a predecir el precio final de cada vivienda.
        \end{frame}
    % Sección 2
    \section{2.- Data}
        \begin{frame}{Data}
        	\begin{itemize}
        		\item La data se obtuvo del siguiente enlace: \textcolor{blue}{\url{https://www.kaggle.com/competitions/house-prices-advanced-regression-techniques/data }}
        		\item Existen 4 archivos, de las cuales se seleccionan dos.
        		\begin{itemize}
        			\item train.csv
        			\item data$\_$description.txt
        		\end{itemize}
        		\item La data \textbf{train.csv} cuenta con 80 variables, de las cuales se escogieron 5.
        		\begin{enumerate}
        			\item \textbf{MSZoning.-} Clasificación general de zonificación.
        			\item \textbf{Condition1.-} Proximidad a carretera principal o vía férrea.
        			\item \textbf{OverallQual.-} Material general y calidad de acabado.
        			\item \textbf{GrLivArea.-} Pies cuadrados de superficie habitable sobre el nivel de suelo.
        			\item \textbf{SalePrice.-} Precio de venta de la propiedad en dólares. Esta es la variable de destino que está tratando de predecir.
        		\end{enumerate}
        	\end{itemize}
        \end{frame}
    % Sección 3
    \section{3.- Preguntas}
	    \begin{frame}{Preguntas}
	    	\begin{enumerate}
	    		\item ¿Cuál es la correlación entre el área habitada y el precio de la vivienda?
	    		\item ¿Cuál es la zona de clasificación con mayor demanda de clientes?
	    		\item ¿Según la proximidad a la carretera principal cual es la categoría que tiene mayor número de ventas?
	    		\item ¿Cuál es la relación que existe entre el acabado de la vivienda y el precio de venta?
	    	\end{enumerate}
	    \end{frame}          
\end{document}